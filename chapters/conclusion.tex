\documentclass[../thesis.tex]{subfiles}

\begin{document}

Σε αυτή τη διπλωματική εργασία παρουσιάσαμε μία ολοκληρωμένη εφαρμογή συνεπιβατισμού η οποία στοχεύει να λύσει το πρόβλημα της συμφόρησης που αναπτύσσεται κοντά στο Πολυτεχνείο.
Η εκπόνηση του παρόντος έργου περιέλαβε την έρευνα και βαθύτερη εξοικείωσή μας όχι μόνο με την ανάπτυξη κινητών εφαρμογών, αλλά και με θέματα που αφορούν την λειτουργία των server και των βάσεων δεδομένων, τη διαδικτυακή επικοινωνία, και την ταυτοποίηση.
Παράλληλα εξοικειωθήκαμε και με συγκεκριμένες τεχνολογίες οι οποίες είτε είναι εδραιωμένες στον τομέα των διαδικτυακών εφαρμογών όπως το Node.js/Express.js και η MySQL, είτε είναι ανερχόμενες όπως το Flutter.

Η εφαρμογής εμφανίζει αρκετά σύνθετη λειτουργία, αφού είναι υπεύθυνη για τον ταυτόχρονο συντονισμό πολλαπλών διαφορετικών διαδρομών στις οποίες εμπλέκονται πολλαπλοί χρήστες, όλα σε πραγματικό χρόνο.
Ωστόσο ευελπιστούμε ότι η εφαρμογή είναι σε θέση να χειριστεί αποτελεσματικά την πολυπλοκότητα αυτή, και να δώσει λύση στο προκείμενο πρόβλημα.
Έπειτα από προκαταρκτικές δοκιμές για να εξασφαλείσουμε ότι η εφαρμογή λειτουργεί όσο το δυνατόν καλύτερα, το επόμενο στάδιο είναι η διετέλεση περισσότερων δοκιμών σε πραγματικές πλεόν συνθήκες μέσω ενός beta test, για τον εντοπισμό σφαλμάτων και την περαιτέρω βελτίωση της εφαρμογής μέσα από το feedback των φοιτητών.
Μετά το πέρας του δοκιμαστικού σταδίου, η εφαρμογή μπορεί να γίνει διαθέσιμη σε όλους πλέον τους φοιτητές της σχολής και να αξιολογηθεί η αποτελεσματικότητά της για τη μείωση του συνωστισμού.
Σε περίπτωση που η εφαρμογή είναι επιτυχής, στο μέλλον θα μπορούσε ακόμα να επεκταθεί και για χρήση σε άλλα πανεπιστήμια τα οποία αντιμετωπίζουν παρόμοιο πρόβλημα.

\end{document}