\documentclass[../thesis.tex]{subfiles}

\begin{document}

Όπως έχουμε αναφέρει, ένα μέρος της εφαρμογής αφορά το ανέβασμα εικόνων από τους χρήστες για εμφάνιση στο δημόσιο προφίλ τους.
Ήταν σημαντικό να προσφέρουμε τη δυνατότητα αυτή στους χρήστες, καθώς θεωρήσαμε ότι οι φωτογραφίες καθιστούν την εφαρμογή πιο εύχρηστη, επιτρέποντας στους χρήστες να αναγνωρίσουν εύκολα ο ένας τον άλλον, ενώ ενισχύουν ταυτόχρονα την κοινωνική διάσταση της εφαρμογής.
Ωστόσο, αυτή η επιλογή που δίνεται στους χρήστες ενδέχεται να παρουσιάσει κινδύνους.
Ο κάθε χρήστης μπορεί να ανεβάσει φωτογραφίες ακατάλληλου περιεχομένου, και οι φωτογραφίες αυτές μπορεί να εμφανιστούν στις συσκευές άλλων χρηστών, γεγονός που πρέπει σε κάθε περίπτωση να αποφευχθεί.
Η επώνυμη φύση της εφαρμογής που επιβάλλεται από την απαίτηση ταυτοποίησης μειώνει σημαντικά την πιθανότητα τέτοιων περιστατικών, όμως πρέπει να είμαστε σε θέση να εντοπίσουμε άμεσα το υλικό αυτό αν τυχόν εμφανιστεί.

\bigskip

Για τον σκοπό αυτό χρησιμοποιήσαμε ένα μοντέλο μηχανικής μάθησης.
Αφού δοκιμάσαμε αρκετά μοντέλα, αποφασίσαμε να επιλέξουμε για την περίπτωσή μας το OpenNSFW2\cite{Yung_Open-NSFW_2_2023}, το οποίο αποτελεί σύγχρονη υλοποίηση του δημοφιλούς μοντέλου OpenNSFW\cite{Mahadeokar_Pesavento_2016} της Yahoo.
Το μοντέλο είναι τύπου Residual Network, και βασίζεται στο ResNet50 της Microsoft, με υλοποίηση στη βιβλιοθήκη Tensorflow.
Επιλέξαμε το συγκεκριμένο μοντέλο καθώς έπειτα από δοκιμές συμπεράναμε ότι προσφέρει τη μεγαλύτερη ακρίβεια με μικρό κόστος τόσο όσον αφορά το μέγεθος όσο και τον υπολογιστικό φόρτο.

\bigskip

Για τη λειτουργία του μοντέλου δημιουργήσαμε μέσω του framework Flask έναν απλό τοπικό server Python\footnote{Αρχικά επιχειρήσαμε να χρησιμοποιήσουμε το Node.js και για αυτόν τον σκοπό ώστε να έχουμε ομοιομορφία όσον αφορά τη γλώσσα, όμως κάποιες ασυνέπειες στην υλοποίηση των διεπαφών του Tensorflow έκαναν προβληματική τη φόρτωση του μοντέλου στη JavaScript} ο οποίος λειτουργεί παράλληλα με τον κύριο server.
Ο server αυτός φορτώνει το μοντέλο μηχανικής όρασης και αναμένει αιτήματα HTTP για την αξιολόγηση του περιεχομένου ενός δεδομένου αρχείου.
Μόλις λάβει ένα αίτημα, ο server δίνει το αρχείο εικόνας ως είσοδο στο μοντέλο και λαμβάνει ως έξοδο την πιθανότητα απεικόνισης ακατάλληλου περιεχομένου, την οποία στέλνει ως απάντηση πίσω στον κύριο server.

Είναι σημαντικό να σημειωθεί ότι εφόσον οι δύο servers ανήκουν σε διαφορετικές διεργασίες η λειτουργία τους είναι παράλληλη, και καθώς η επικοινωνία επιτυγχάνεται μέσω ασύγχρονων non-blocking HTTP calls ο κύριος server σε καμία περίπτωση δεν παύει να επεξεργάζεται αιτήματα περιμένοντας την απόκριση του server μηχανικής όρασης.
Η υπολογιστικά ακριβή εργασία του classification γίνεται σε ξεχωριστό thread, και ακόμα και σε διαφορετικό πυρήνα CPU.
Έτσι μπορούμε να έχουμε γρήγορη αναγνώριση εικόνων και επομένως άμεση εντόπιση πιθανών παραβάσεων, χωρίς όμως αυτό να είναι εις βάρος της ταχύτητας του server και επομένως της εμπειρίας του χρήστη.

\end{document}