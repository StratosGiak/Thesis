\documentclass[../thesis.tex]{subfiles}

\begin{document}

\section{Το πρόβλημα}
Το πρόβλημα το οποίο καλούμαστε να λύσουμε αφορά τον συνωστισμό που παρατηρείται συχνά στη στάση λεωφορείου του \textit{Στρατιωτικού Νοσοκομείου} που βρίσκεται στη λεωφόρο Κατεχάκης, κοντά στη σχολή του Ε.Μ.Π.
Ένα σημαντικό ποσοστό των φοιτητών προσέρχεται καθημερινά στη σχολή χρησιμοποιώντας τα μέσα μαζικής μεταφοράς, και συγκεκριμένα το λεωφορείο 242 που μεταφέρει άτομα από τη στάση απευθείας στη σχολή.
Η στάση κατέχει κομβική θέση λόγω του μεγάλου αριθμού φοιτητών που τη χρησιμοποιούν, ωστόσο το γεγονός αυτό έχει ως αποτέλεσμα την αδυναμία επαρκούς κάλυψης των αναγκών των επιβατών από τα δρομολόγια των λεωφορείων, και επομένως τη συγκέντρωση πλήθους στην περιοχή.

Η συγκέντρωση πλήθους με τη σειρά της σημαίνει ότι οι φοιτητές αναγκάζονται να σταθούν εκτός του πεζοδρομίου και εντός της λεωφόρου, διαταράσσοντας την κυκλοφορία και θέτοντας την ασφάλειά τους σε κίνδυνο.
Εκτός αυτού, πολλές φορές ο αριθμός φοιτητών είναι αρκετά μεγάλος ώστε να υπερβαίνει τη χωρητικότητα του κάθε λεωφορείου που φτάνει στη στάση, με αποτέλεσμα οι επιβαίνοντες στο λεωφορείου να συνωστίζονται σε επικίνδυνο βαθμό, και οι φοιτητές που αδυνατούν να επιβιβαστούν να χρειάζεται να αναμένουν το επόμενο λεωφορείο.
Η κατάσταση είναι χειρότερη κατά τις ώρες αιχμής, ενώ δυσχεραίνεται περαιτέρω όταν παρατηρείται κυκλοφοριακή συμφόρηση ή καθυστερήσεις στα δρομολόγια, αλλά και κατά τη διάρκεια της εξεταστικής περιόδου όπου η προσέλευση των φοιτητών είναι αυξημένη.

Ταυτόχρονα με τους φοιτητές που μεταβαίνουν στη σχολή μέσω των ΜΜΜ, ένας αριθμός φοιτητών χρησιμοποιεί οχήματα ΙΧ για τη μεταφορά του.
Έχει παρατηρηθεί επομένως το φαινόμενο οι φοιτητές αυτοί να σταματάνε το όχημά τους στη στάση για να επιτρέψουν σε συμφοιτητές τους να εισέλθουν στο όχημα και να μεταβούν μαζί στη σχολή, μειώνοντας κατά ένα μικρό βαθμό τον συνωστισμό.

Η εφαρμογή που αναπτύξαμε έχει στόχο την αξιοποίηση αυτής ακριβώς της συμπεριφοράς προκειμένου να καταπολεμηθεί πιο στοχευμένα και αποτελεσματικά το παραπάνω πρόβλημα.
Συγκεκριμένα η εφαρμογή συντονίζει το ταίριασμα των εν κινήσει οδηγών με τους πεζούς που αναμένουν στη στάση, προσφέροντας με αυτόν τον τρόπο ένα πιο οργανωμένο πλαίσιο στο οποίο μπορεί να συμβεί ο συνεπιβατισμός ο οποίος έχει αναπτυχθεί οργανικά στις δεδομένες συνθήκες.
Παράλληλα, η εφαρμογή συμβάλλει στην ανάπτυξη διαπροσωπικών σχέσεων μεταξύ των φοιτητών, καθώς φέρνει σε επαφή άτομα τα οποία δεν γνωρίζονται απαραίτητα μεταξύ τους, ενισχύοντας έτσι το κλίμα αλληλεγγύης και συναδελφικότητας στη σχολή.

\section{Συνεπιβατισμός}
Ο όρος \textit{συνεπιβατισμός} (carpooling στα Αγγλικά) αναφέρεται στη χρήση ενός οχήματος ΙΧ από πολλά άτομα για την άφιξη σε έναν κοινό προορισμό.
Στην πιο συνηθισμένη περίπτωση, ο συνεπιβατισμός περιλαμβάνει την εκ των προτέρων συνεννόηση μεταξύ του επιβάτη και του οδηγού πριν την έναρξη της διαδρομής, προκειμένου να αποφασιστεί ο ακριβής προορισμός, η ώρα συνάντησης, και το σημείο επιβίβασης.
Επομένως, αυτό το είδος συνεπιβατισμού εν γένει χαρακτηρίζεται από μειωμένη ευελιξία, και περιορίζεται κυρίως σε φίλους, ή συνεργάτες.

\bigskip

Μία άλλη μορφή συνεπιβατισμού αναπτύχθηκε στη Washington DC τη δεκαετία του 1970 και έπειτα εξαπλώθηκε και σε άλλες πόλεις της Αμερικής και του κόσμου, το λεγόμενο casual carpooling ή ``slugging''.

Σε ορισμένους αυτοκινητοδρόμους των πόλεων αυτών τα οχήματα ΙΧ με περισσότερους από έναν επιβάτη είχαν πρόσβαση σε λωρίδες ταχείας κυκλοφορίας και μειωμένα τέλη διοδίων, γεγονός που έδινε κίνητρο σε οδηγούς να δέχονται πεζούς στο όχημά τους και να τους μεταφέρουν στον προορισμό τους\cite{SHAHEEN2016165}.
Καθώς η δραστηριότητα αυτή ωφελούσε αμφότερους πεζούς και οδηγούς, μειώνοντας εκτός των άλλων το χρόνο μετακίνησης και για τους δύο, το casual carpooling άνθισε στις πόλεις αυτές, με τη γρήγορη καθιέρωση ημιεπίσημων ``στάσεων'' όπου οι πεζοί ακόμη και σήμερα μπορούν να αναμένουν τους διερχόμενους οδηγούς.
Αυτό το είδος συνεπιβατισμού δίνει τη δυνατότητα σε άτομα άγνωστα μεταξύ τους να μοιραστούν το όχημα άμεσα και χωρίς την ανάγκη προηγούμενης συνεννόησης, ωστόσο είναι περιορισμένο ως προς τα σημεία συνάντησης, καθώς αυτά είναι λιγοστά και καθορισμένα.

\bigskip

Με την έλευση των κινητών εφαρμογών η δημοφιλία του συνεπιβατισμού αυξήθηκε ραγδαία, μέσω των λεγόμενων ride-hailing apps.

Μέσω των εφαρμογών αυτών οι χρήστες έχουν πλέον τη δυνατότητα να αναζητήσουν επί τόπου οδηγό, οπουδήποτε και αν βρίσκονται.
Οι εφαρμογές παρουσιάζουν όλους τους κοντινούς διαθέσιμους οδηγούς στον χρήστη, ο οποίος έχει την ελευθερία να επιλέξει όχι μόνο τον οδηγό που θα τον παραλάβει, αλλά και τον ακριβή προορισμό του, αίροντας πρακτικά όλους τους περιορισμούς που αναφέρθηκαν προηγουμένως.
Ωστόσο, οι ανέσεις αυτές που προσφέρονται στον πεζό επιβάλλουν συχνά τη χρέωση της υπηρεσίας, η οποία ουσιαστικά μετατρέπεται σε υπηρεσία ταξί.
Πράγματι, σχεδόν όλες οι διαθέσιμες εφαρμογές ride-hailing είναι εμπορικές.
Η πρώτη εμπορική εταιρεία που εισήλθε στον τομέα των ride-hailing εφαρμογών ήταν η Uber, η οποία το 2010 δημιούργησε την επώνυμη εφαρμογή της, και η οποία διατηρεί μέχρι και σήμερα το μεγαλύτερο μερίδιο της αγοράς σε παγκόσμιο επίπεδο.

\section{Η λειτουργία της εφαρμογής}

Η εφαρμογή που αναπτύξαμε πλησιάζει περισσότερο στη φιλοσοφία της το casual carpooling, αλλά ενσωματώνει και χαρακτηριστικά από τις εφαρμογές ride-hailing.

Το σημείο συνάντησης καθώς και ο τελικός προορισμός είναι καθορισμένα: ο πεζός πρέπει να μεταβεί στη στάση του Στρατιωτικού Νοσοκομείου για να μπορέσει να αναζητήσει οδηγούς μέσω της εφαρμογής, ενώ ο προορισμός είναι προφανώς η σχολή του πολυτεχνείου.
Μόλις ένας οδηγός ενεργοποιήσει την εφαρμογή και εισέλθει εντός μίας ακτίνας μερικών χιλιομέτρων από τη στάση, ενημερώνεται για την ύπαρξη πεζών που περιμένουν, και δηλώνει την πρόθεση του να τους δεχτεί στο όχημά του.
Οι πεζοί με τη σειρά τους ενημερώνονται άμεσα μόλις κάποιος οδηγός δεχτεί να τους παραλάβει, και ``κλειδώνουν'' τη θέση τους στο όχημα.
Το μήνυμα για τον διαθέσιμο οδηγό στέλνεται ταυτόχρονα σε πολλαπλούς πεζούς, και η αντίστοιχη θέση παραδίδεται σε εκείνον που απάντησε πιο γρήγορα στην ειδοποίηση.

Ύστερα από την επιτυχή αντιστοίχιση πεζού και οδηγού, ο κάθε χρήστης παρουσιάζεται με τα στοιχεία του άλλου ώστε να είναι δυνατή η αναγνώριση κατά τη συνάντηση (όνομα, φωτογραφίες, αριθμός κυκλοφορίας), ενώ παράλληλα παρουσιάζεται το στίγμα του στο χάρτη για τον προσδιορισμό της θέσης και του χρόνου άφιξης.
Μόλις ο οδηγός παραλάβει τους πεζούς από τη στάση κατευθύνονται μαζί προς τη σχολή, και κατά την άφιξή τους σε αυτή, η εφαρμογή τερματίζει αυτόματα τη συνεδρία.
Τέλος, παρέχεται η ευκαιρία αξιολόγησης τόσο από πλευράς του οδηγού όσο και από πλευράς του επιβάτη.

\bigskip

Όλος ο κώδικας για την εφαρμογή είναι διαθέσιμος στο GitHub.
Ο κώδικας της εφαρμογής κινητού βρίσκεται στον σύνδεσμο \cite{app_code}, και ο κώδικας του server βρίσκεται στον σύνδεσμο \cite{server_code}.

\end{document}