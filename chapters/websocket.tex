\documentclass[../thesis.tex]{subfiles}

\begin{document}

Το WebSocket είναι ένα πρωτόκολλο επικοινωνίας το οποίο επιτρέπει την ανταλλαγή μηνυμάτων μεταξύ client και host και προς τις δύο κατευθύνσεις ταυτόχρονα (full duplex).
Το WebSocket αποτελεί συγκεκριμένα ενα application-layer protocol σύμφωνα με την ορολογία του μοντέλου OSI, το οποίο χρησιμοποιεί το TCP ως transport layer για τη μεταφορά των frames που περιέχουν τα δεδομένα προς αποστολή.

\bigskip

Μία πρώιμη εκδοχή του WebSocket εμφανίστηκε για πρώτη φορά το 2008 σε ένα προσχέδιο των προδιαγραφών του HTML5 υπό το όνομα TCPConnection, ενώ η τελική έκδοση του πρωτοκόλλου δημοσιεύτηκε το 2011.
Πριν από το WebSocket, οι επιλογές για duplex επικοινωνία client και server ήταν περιορισμένες, και βασίζονταν συχνά στην κατάχρηση του πρωτοκόλλου HTTP.
Για παράδειγμα, μία μέθοδος που χρησιμοποιούταν πριν τη διάδοση των WebSocket για αμφίδρομη επικοινωνία με έναν server ήταν το λεγόμενο \textit{long polling}.
Σύμφωνα με την τακτική του long polling ο χρήστης στέλνει κανονικά HTTP requests στον server, αλλά αντίθετα με τη συνηθισμένη περίπτωση όπου ο server στέλνει αμέσως το response στο χρήστη, ο server αναμένει, αφήνοντας τη σύνδεση ανοιχτή, και στέλνοντας το αντίστοιχο response μόνο μόλις εμφανιστούν νέα δεδομένα.
Τη στιγμή που ο χρήστης λάβει το response αυτό η σύνδεση τερματίζεται, και ο χρήστης στέλνει αμέσως ένα νέο request, συνεχίζοντας την ίδια διαδικασία.

\bigskip

Παρόλο που με αυτόν τον τρόπο επιτυγχάνεται η συνεχής επικοινωνία του server με τον client, το πρωτόκολλο HTTP δεν είναι σχεδιασμένο για αυτόν τον σκοπό, και μία τέτοια υλοποίηση δεν είναι ιδανική.
Το long polling στις περισσότερες περιπτώσεις καταναλώνει πολύ περισσότερους πόρους από ότι θα χρειαζόταν μία απλή σύνδεση, ενώ εισάγονται μεγάλες καθυστερήσεις λόγω του overhead που απαιτεί το HTTP.
Εξ'άλλου, το long polling προορίζεται περισσότερο για περιπτώσεις όπου ο μεγαλύτερος όγκος της επικοινωνίας συμβαίνει στην κατεύθυνση server προς client.
Παρόμοιες λύσεις που βασίζονται στο HTTP όπως τα \textit{Server Sent Events} λύνουν κάποια από τα προβλήματα του long polling, αλλά εξακολουθούν να διαθέτουν αυτήν την ασυμμετρία ανάμεσα στα δύο άκρα της σύνδεσης.

\bigskip

Το WebSocket σχεδιάστηκε εξ αρχής ως πρωτόκολλο προορισμένο για full duplex συνδέσεις, και επομένως υπερέχει των παραπάνω εναλλακτικών όσον αφορά την real-time αμφίδρομη επικοινωνία client και server.
Είναι σημαντικό να αναφέρουμε ότι όλες οι παραπάνω μέθοδοι (συμπεριλαμβανομένων των WebSocket) σχεδιάστηκαν κυρίως για χρήση σε browsers, και επομένως οι περιορισμοί που αναφέραμε είναι απόρροια των περιορισμών που επιβάλλουν οι browsers.

Ένα εναλλακτικό πρωτόκολλο το οποίο είναι κατάλληλο για αμφίδρομη real-time επικοινωνία είναι το WebRTC.
Το WebRTC είναι σχεδιασμένο για real-time \textit{communication}, δηλαδή για την επικοινωνία χρηστών μέσω audio και video streaming, και για τον σκοπό αυτό χρησιμοποιεί το πρωτόκολλο UDP.
Το UDP επιτυγχάνει το χαμηλό latency το οποίο είναι απαραίτητο για την ομαλή οπτικοακουστική επικοινωνία, κάνοντας όμως αρκετές υποχωρήσεις όσον αφορά την αξιοπιστεία των πακέτων δεδομένων.
Το UDP συγκεκριμένα δεν προσφέρει καμία εγγύηση για την ακεραιότητα των πακέτων, αλλά ούτε και για την σωστή σειρά λήψης τους από τον δέκτη.

Αυτό το γεγονός καθιστά το πρωτόκολλο ακατάλληλο για την περίπτωσή μας,
δεδομένου ότι επιθυμούμε να είμαστε σίγουροι ότι όλα τα μηνύματα φτάνουν από τους χρήστες στον server στη σωστή σειρά, χωρίς σφάλματα, επαναλήψεις, ή απώλειες.
Το πρωτόκολλο το οποίο προσφέρει τις εγγυήσεις αυτές είναι το TCP, και πάνω σε αυτό το πρωτόκολλο βασίζεται το WebSocket, γεγονός που δικαιολογεί την επιλογή μας.

\bigskip

Για την επίτευξη αμφίδρομης επικοινωνίας μεταξύ δύο αυθαίρετων χρηστών, μία απλή σύνδεση TCP είναι σαφώς αρκετή, και μάλιστα είναι πιο αποδοτική σε σχέση με την προσθήκη του WebSocket.
Πράγματι, καθώς η εφαρμογή μας είναι native και δεν κάνει χρήση του browser, θα μπορούσαμε να υλοποιήσουμε την επικοινωνία χρηστών και server μέσω απλών TCP sockets, χωρίς τη χρήση των WebSocket.
Ωστόσο, το TCP είναι ένα stream-based πρωτόκολλο, γεγονός που σημαίνει ότι τα δεδομένα φτάνουν στον δέκτη ως μία συνεχή ροή από bytes, χωρίς διαχωρισμό μεταξύ των μηνυμάτων που στέλνει ο αποστολέας.
Με άλλα λόγια κάθε φορά που ο δέκτης διαβάζει τα δεδομένα από το socket, δεν λαμβάνει απαραίτητα ένα μήνυμα όπως αυτό στάλθηκε από τον πομπό, αλλά ενδεχομένως να λάβει μόνο μέρος ενός μηνύματος, ή ακόμα και πολλαπλά μηνύματα μαζί.

Έτσι, το πρωτόκολλο TCP χρειάζεται να επαυξηθεί με το κατάλληλο \textit{framing} των δεδομένων, το οποίο θα παρουσιάζει στον δέκτη μεμονωμένα μηνύματα αντί για ένα byte stream.
Το πρωτόκολλο WebSocket κάνει ακριβώς αυτό, προσθέτοντας απλά έναν μηχανισμό framing πάνω στο TCP με πολύ μικρό overhead\cite{rfc6455}.
Για χάρη ευκολίας λοιπόν επιλέξαμε να κάνουμε χρήση του WebSocket στην εφαρμογή μας, αντί να υλοποιήσουμε το δικό μας πρωτόκολλο framing.


\end{document}