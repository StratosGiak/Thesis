% !TEX root = ../thesis.tex

\documentclass[../thesis.tex]{subfiles}

\begin{document}

Το Flutter είναι ένα framework ανάπτυξης cross-platform εφαρμογών για Android, iOS και Web που δημιουργήθηκε και αναπτύσσεται από την Google.
Η Google παρουσίασε για πρώτη φορά το 2015 μία πειραματική έκδοση του Flutter υπό την ονομασία “Sky”, αναφέροντας ως πρωταρχικό στόχο του νέου αυτού framework την εύκολη και γρήγορη ανάπτυξη αποδοτικών mobile εφαρμογών.
Το Flutter είναι άρρηκτα συνδεδεμένο με την γλώσσα προγραμματισμού Dart, η οποία αναπτύσσεται επίσης από την Google.
Ως framework ανάπτυξης cross-platform εφαρμογών, το Flutter καθιστά δυνατή την συγγραφή ενιαίου κώδικα για όλες τις υποστηριζόμενες πλατφόρμες.
Ο προγραμματιστής αρκεί να αναπτύξει την εφαρμογή μία φορά στη γλώσσα Dart,
αγνοώντας σε μεγάλο βαθμό τις λεπτομέρειες υλοποίησης της κάθε πλατφόρμας,
και όλες οι διαδικασίες που είναι απαραίτητες για την λειτουργία της εφαρμογής στη εκάστοτε πλατφόρμα γίνονται αυτόματα.
Συγκεκριμένα, μέσω του Flutter SDK ο κώδικας Dart που περιλαμβάνει την λογική και το UI της εφαρμογής μεταγλωττίζεται σε native machine code (πχ. ARM για κινητές συσκευές)

\section{Cross-platform}
Η cross-platform λειτουργία επιτυγχάνεται εν μέρει μέσα από την συμπερίληψη στο Flutter ειδικού rendering engine (Skia/Impeller) ο οποίος χειρίζεται εξ ολοκλήρου την σωστή εμφάνιση των στοιχείων του UI ανεξαρτήτως της πλατφόρμας.
Επομένως οι εφαρμογές Flutter δεν χρησιμοποιούν τα native UI components του κάθε λογισμικού συστήματος, αλλά αντιθέτως «ζωγραφίζουν» τα δικά τους components πάνω σε έναν καμβά, pixel ανά pixel.
Αυτή η μέθοδος απεικόνισης προσφέρει περισσότερη ομοιομορφία στην εμφάνιση των εφαρμογών σε όλες τις πλατφόρμες, ξεχωρίζοντας το Flutter από άλλα frameworks όπως το React Native (το οποίο αποτελεί τον κύριο ανταγωνιστή του Flutter στον τομέα των cross-platform εφαρμογών).

\section{Widgets}
Η βασική μονάδα όλων των στοιχείων του UI στο Flutter είναι το widget.
Ένα widget περιγράφει ένα μέρος του UI της εφαρμογής, όπως ένα κουμπί, κείμενο, ή εικονίδιο.
Πέρα από τα εμφανή λειτουργικά στοιχεία, στα widgets συμπεριλαμβάνονται και στοιχεία που ελέγχουν την διάταξη/layout της εφαρμογής, όπως τα widget στοίχισης, padding, και δημιουργίας στηλών/γραμμών.
Η εφαρμογή έχει ως ρίζα ένα βασικό widget το οποίο αποτελεί το υπόβαθρο του συνόλου της διεπαφής, και όλα τα στοιχεία του UI δομούνται μέσα από την σύνθεση και την εμφώλευση απλών widget.
Έτσι όλες οι εφαρμογές Flutter έχουν την δομή ενός δέντρου, με το κάθε widget να διαθέτει έναν γονιό/parent (το widget μέσα στο οποίο είναι εμφωλευμένο) και ενδεχόμενα παιδιά/children (τα widget τα οποία περιέχει).

\end{document}