\documentclass[a4paper, 11pt]{report}
\usepackage{graphicx}
\usepackage[greek]{babel}
\usepackage[Greek, Latin]{ucharclasses}
\usepackage{xltxtra, xgreek}
\usepackage{xcolor}
\usepackage{tcolorbox}
\usepackage{subfiles}
\usepackage[sorting=none]{biblatex}
\usepackage{hyperref}
\usepackage{afterpage}
\usepackage{setspace}
\onehalfspacing
\setTransitionsForGreek{\setlanguage{greek}}{\setlanguage{american}}
\setmainfont[Kerning=On,Mapping=tex-text]{Linux Libertine O}
\graphicspath{{images/}}
\addbibresource{bibliography.bib}
\definecolor{bg}{gray}{0.95}
\tcbset{colback=bg}
\tcbuselibrary{minted}
\tcbuselibrary{breakable}
\newtcblisting{codeblock}[3][]{
    left=6.5mm,
    title=#3,
    breakable,
    minted language=#2,
    minted options={linenos, autogobble, breaklines, fontsize=\footnotesize, baselinestretch=1}
    #1,
    listing only
    }
\usepackage{csquotes}
\usepackage{subcaption}
\captionsetup[subfigure]{
    font=footnotesize,
    labelfont=up,
    justification=centering
    }
\newcommand\blankpage{%
    \null
    \thispagestyle{empty}%
    \newpage
    }

\title{
    {Διαδικτυακή εφαρμογή για τον δυναμικό συνεπιβατισμό μεταφορικών μέσων}
}
\author{Ευστράτιος Γιακουμάκης}
\date{10η Οκτωβρίου 2024}

\begin{document}

\maketitle

\begin{abstract}\thispagestyle{plain}
    Εδώ και πολλά χρόνια έχει παρατηρηθεί το φαινόμενο οι φοιτητές που προσέρχονται στην Πολυτεχνειούπολη του Εθνικού Μετσοβίου Πολυτεχνείου μέσω των συγκοινωνιών να συγκεντρώνονται σε ορισμένες κομβικές στάσεις, οδηγώντας σε συνωστισμό και διαταράσσοντας τη γύρω κυκλοφορία.
    Αρκετοί ωστόσο φοιτητές χρησιμοποιούν το ιδιωτικό τους όχημα για τη μετακίνησή του στο Πολυτεχνείο, και εφόσον πορεύονται κατά μήκος των συγκεκριμένων οδών εμφανίζεται η ευκαιρία επιβίβασης στο ΙΧ κάποιων εκ των φοιτητών που περιμένουν στη στάση, ελαφρύνοντας έτσι τη συμφόρηση στον χώρο.
    Στην εργασία αυτή θα παρουσιάσουμε μία κινητή εφαρμογή η οποία καθιστά δυνατή την επικοινωνία των φοιτητών που βρίσκονται στο όχημά τους με τους φοιτητές που βρίσκονται στη στάση, αναλαμβάνοντας το ταίριασμα οδηγών και πεζών, ενθαρρύνοντας τον συνεπιβατισμό, και συμβάλλοντας έτσι στην επίλυση του προβλήματος.
    Για την ανάπτυξη της εφαρμογής χρησιμοποιήσαμε ένα tech stack που περιλαμβάνει το framework του Flutter στο frontend, και τον συνδυασμό ενός server Node.js με μία βάση δεδομένων MySQL στο backend, με την επικοινωνία των δύο μερών να συντονίζεται μέσω του πρωτοκόλλου WebSocket.
    Η εφαρμογή έχει δοκιμαστεί ενδελεχώς σε κλειστό περιβάλλον, και έχοντας διαπιστωθεί ότι λειτουργεί ικανοποιητικά, είναι πλέον έτοιμη για πιλοτική δοκιμή σε πραγματικές συνθήκες με μετέπειτα deployment στις υπηρεσίες του App Store και Google Play.
    
    \hspace{10pt}

    \small	
    \textbf{\textit{Λέξεις-κλειδιά---}}
    carpooling, ride-hailing, mobile app, Flutter, Node.js
    \afterpage{\blankpage}
\end{abstract}

\section*{Ευχαριστίες}
Ευχαριστώ τον καθηγητή και κοσμήτορα της σχολής κ. Παναγιώτη Τσανάκα για την ευκαιρία που μου έδωσε μέσω αυτής της διπλωματικής, καθώς και για τη καθοδήγηση που μου προσέφερε καθ' όλη τη διάρκεια ανάπτυξης της εφαρμογής. Επίσης ευχαριστώ τον Κωνσταντίνο Αθανασιάδη για την τεχνική υποστήριξη και διαχείρηση λοιπών θεμάτων και τον Άγγελο Κολαΐτη για την διαχείρηση του server και της υπηρεσίας ταυτοποίησης. Τέλος, ευχαριστώ την οικογένειά μου και τους φίλους μου, που ήταν πάντα στο πλευρό μου.

\tableofcontents

\afterpage{\blankpage}

\chapter{Εισαγωγή}
\subfile{chapters/intro.tex}

\afterpage{\blankpage}

\chapter[Flutter]{Flutter\footnote{Πηγή για τις πληροφορίες του κεφαλαίου αποτελούν τα επίσημα docs του Flutter \cite{flutter_architecture, flutter_inside} εκτός όπου αναγράφεται διαφορετικά}}
\subfile{chapters/flutter.tex}

\afterpage{\blankpage}

\chapter{Node.js}
\subfile{chapters/node.tex}

\afterpage{\blankpage}

\chapter{WebSocket}
\subfile{chapters/websocket.tex}

\afterpage{\blankpage}

\chapter{Frontend}
\subfile{chapters/app.tex}

\afterpage{\blankpage}

\chapter{Backend}
\subfile{chapters/server.tex}

\afterpage{\blankpage}

\chapter{Συμπεράσματα}
\subfile{chapters/conclusion.tex}

\afterpage{\blankpage}

\appendix
\subfile{chapters/appendix.tex}

\afterpage{\blankpage}

\printbibliography

\end{document}